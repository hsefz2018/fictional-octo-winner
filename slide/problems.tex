\documentclass[UTF8]{beamer}

\usetheme{Madrid}
\usecolortheme{seahorse}

\usepackage[english]{babel}
\usepackage{ctex}
\usepackage{algorithm,algorithmic}
\usepackage{tabularx}

\begin{document}

\title{SHTSC 2016 题目原文}
\frame{\titlepage}

%%%%%%%%%%%%%%%%%%%%%%%% Day 1 A %%%%%%%%%%%%%%%%%%%%%%%%

\begin{frame}{Problem 1. Robot}
[Touhou Contest by Nettle, Stage 1 A]

在幻想乡,东风谷早苗是以高达控闻名的高中生宅巫女。某一天,早苗终于入手了最新款的
钢达姆模型。作为最新的钢达姆,当然有了与以往不同的功能了,那就是它能够自动行走,厉害吧。

早苗的新模型可以按照输入的命令进行移动,命令包含“E”、“S”、“W”、“N”四种,
分别对应四个不同的方向,依次为东、南、西、北。执行某个命令时,它会向着对应
方向移动一个单位。作为新型机器人,自然不会只单单执行一个命令,它可以执行命令串。
对于输入的命令串,每一秒它会按照命令行动一次。而执行完命令串最后一个命令后,会自
动从头开始循环。在 0 时刻时早苗将钢达姆放置在了 (0,0) 的位置,并且输入了命令串。
她想要知道 $T$ 秒后钢达姆所在的位置坐标。

\end{frame}

\begin{frame}{Problem 1. Robot}

\textbf{输入}
\begin{itemize}
    \item 第 1 行:命令串 $S$
    \item 第 2 行:正整数 $T$
\end{itemize}
\textbf{输出}
\begin{itemize}
    \item 第 1 行:结束时的位置 $(X, Y)$
\end{itemize}

\begin{tabularx}{\textwidth}{|X|X|}
\hline
\texttt{\textbf{robot.in}} & \texttt{\textbf{robot.out}} \\ \hline
\texttt{NSWWNSNEEWN}\newline
\texttt{12}
&
\texttt{-1 3}
\\ \hline
\end{tabularx}
\newline
\begin{itemize}
    \item 60 pts: $|T| \leq 500\,000, |S| \leq 5\,000$
    \item 100 pts: $|T| \leq 2\,000\,000\,000, |S| \leq 5\,000$
\end{itemize}

\end{frame}

%%%%%%%%%%%%%%%%%%%%%%%% Day 1 B %%%%%%%%%%%%%%%%%%%%%%%%

\begin{frame}{Problem 2. Ice}
[Touhou Contest by Nettle, Stage 1 C]

在幻想乡,琪露诺是以笨蛋闻名的冰之妖精。某一天,琪露诺又在玩速冻青蛙,就是用冰把
青蛙瞬间冻起来。但是这只青蛙比以往的要聪明许多,在琪露诺来之前就已经跑到了河的对
岸。于是琪露诺决定到河岸去追青蛙。

\end{frame}

\begin{frame}{Problem 2. Ice}

小河可以看作一列格子依次编号为 0 到 $N$,琪露诺只能从编号小的格子移动到编号大的格子。
而且琪露诺按照一种特殊的方式进行移动,当她在格子 $i$ 时,她只会移动到 $i+L$ 到 $i+R$
中的一格。你问为什么她这么移动,这还不简单,因为她是笨蛋啊。每一个格子都有一个冰冻指数
$A_i$,编号为 0 的格子冰冻指数为 0 。当琪露诺停留在那一格时就可以得到那一格的冰冻指数
$A_i$。琪露诺希望能够在到达对岸时,获取最大的冰冻指数,这样她才能狠狠地教训那只青蛙。
但是由于她实在是太笨了,所以她决定拜托你帮它决定怎样前进。

开始时,琪露诺在编号 0 的格子上,只要她下一步的位置编号大于 $N$ 就算到达对岸。

\end{frame}

\begin{frame}{Problem 2. Ice}

\textbf{输入}
\begin{itemize}
    \item 第 1 行:三个整数 $N$,$L$,$R$
    \item 第 2 行:$N+1$ 个整数 $A_{0 \dots N}$
\end{itemize}
\textbf{输出}
\begin{itemize}
    \item 第 1 行:最大的冰冻指数
\end{itemize}

\begin{tabularx}{\textwidth}{|X|X|}
\hline
\texttt{\textbf{ice.in}} & \texttt{\textbf{ice.out}} \\ \hline
\texttt{5 2 3}\newline
\texttt{0 12 3 11 7 -2}
&
\texttt{11}
\\ \hline
\end{tabularx}
\newline
\begin{itemize}
    \item 60 pts: $N \leq 10\,000$
    \item 100 pts: $N \leq 200\,000, |A_i| \leq 1000, 1 \leq L \leq R \leq N$
\end{itemize}

\end{frame}

%%%%%%%%%%%%%%%%%%%%%%%% Day 1 C %%%%%%%%%%%%%%%%%%%%%%%%

\begin{frame}{Problem 3. Desired}

[BZOJ 3143] [HNOI 2013]

一个无向连通图,顶点从 1 编号到 $N$,边从 1 编号到 $M$。

小Z在该图上进行随机游走,初始时小Z在 1 号顶点,每一步小Z以相等的概率随机选择当前顶点的某条边,
沿着这条边走到下一个顶点,获得等于这条边的编号的分数。当小Z到达 $N$ 号顶点时游走结束,总分为所有获得的分数之和。
现在,请你对这 $M$ 条边进行编号,使得小Z获得的总分的期望值最小。

\end{frame}

\begin{frame}{Problem 3. Desired}

\textbf{输入}
\begin{itemize}
    \item 第 1 行:$N$,$M$
    \item 接下来 $M$ 行:每行描述一条无向边 $u, v$。
\end{itemize}
\textbf{输出}
\begin{itemize}
    \item 第 1 行:最小的期望值,保留 3 位小数
\end{itemize}

\begin{tabularx}{\textwidth}{|X|X|}
\hline
\texttt{\textbf{desired.in}} & \texttt{\textbf{desired.out}} \\ \hline
\texttt{3 3}\newline
\texttt{2 3}\newline
\texttt{1 2}\newline
\texttt{1 3}
&
\texttt{3.333}
\\ \hline
\end{tabularx}
\newline
\begin{itemize}
    \item 30 pts: $2 \leq N \leq 10$
    \item 100 pts: $2 \leq N \leq 500$
\end{itemize}

\end{frame}

\end{document}
